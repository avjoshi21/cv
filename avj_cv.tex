\documentclass[12pt]{article}
\usepackage{etaremune}
\usepackage[left=1in,right=1in,top=1in,bottom=1in]{geometry}
% \usepackage[pdfversion=2.0,pdfpagemode=UseNone,colorlinks=true]{hyperref}
\usepackage[pdfversion=2.0,pdfpagemode=UseNone,colorlinks=false]{hyperref}
\usepackage{microtype}
\usepackage{parskip}
\usepackage{titlesec}
\usepackage{xcolor}

\definecolor{arxivcolor}{RGB}{179,27,27}
\definecolor{linkcolor}{RGB}{0,70,0}
\hypersetup{urlcolor=linkcolor}

\titleformat{\section}{\large\scshape}{\thesection}{1em}{\vspace{-0.5em}}[\vspace{0.2em}\hrule\vspace{-0.2em}]

\renewcommand\labelitemi{--}

\pagenumbering{gobble}

% \def\withcourses{}
\def\withpress{}

\def\adslibrary{https://ui.adsabs.harvard.edu/public-libraries/d7O0jq6KSTqnvppvvHjpjQ}
\def\googlescholar{https://scholar.google.com/citations?user=Sk6VdjUAAAAJ&hl=en}

\begin{document}

\begin{center}
{\Large Abhishek V. Joshi}\\\vspace{0.5em}
avjoshi2@illinois.edu $|$ \href{https://avjoshi21.github.io}{avjoshi21.github.io} \\\vspace{0.25em} 241 Loomis Laboratory, 1110 W Green St, Urbana, Illinois
\end{center}

\section*{Education}
PhD, Physics, University of Illinois Urbana-Champaign \hfill 2019--present\\
Advisor: Professor Charles Gammie  %\hfill {\small GPA: $3.9/4.0$}


BS, Engineering Physics, University of Illinois Urbana-Champaign \hfill 2015--2018\\
{\small Summa Cum Laude, Computer Science Minor}% \hfill GPA: $3.83/4.0$}

\section*{Research Overview}
I am a computational astrophysicist working on accretion onto supermassive black holes, with a focus on radiative processes that are observationally relevant. This involves applying radiative transfer methods to magnetohydrodynamic models within the context of general relativity. Other topics I am interested in are magnetohydrodynamics, numerical relativity and astrophysical plasmas in general. I use both high-performance computing (HPC) and high-throughput computing (HTC) methods to generate simulations and compare them to observations for model inference.

\section*{Research and Work Experience}
Graduate Research Assistant, UIUC \hfill 2020--2024\\
\indentpar{Worked on both individual projects and projects within the Event Horizon Telescope (EHT) collaboration by applying analytic and numerical models of radiative processes on supermassive black hole accretion flows.}

Graduate Teaching Assistant, UIUC \hfill 2019--2020, Fall 2022, Fall 2024

Software Intern at Thermo Fisher Scientific San Jose, CA \hfill Summer 2019\\
\indentpar{Developed methods for real-time optimization of ion source inlet sprays for mass spectrometers (Patent \href{https://patentcenter.uspto.gov/applications/17197986}{US11791144B2}).}


\section*{Awards and Honors}
University Fellowship Award for Research Accomplishments \hfill Fall 2023\\
Gregory, J.M \& L.C. scholarship \hfill 2017--2018\\
UIUC James Scholar Honors \hfill 2016--2018

\section*{Selected Publications}
Lists: \href{\adslibrary}{NASA ADS}, \href{\googlescholar}{Google Scholar}, \href{https://orcid.org/0000-0002-2514-5965}{ORCID}.
% 2 first-author, 999 citations\\
% Underline indicates mentored student

\begin{etaremune}[leftmargin=1.25em]
\item \textbf{Joshi, A. V.}, Prather, B. S., Chan, C., Wielgus, M., et al. 2024, \href{https://ui.adsabs.harvard.edu/abs/2024ApJ...972..135J}{ApJ, 972, 135}\\\textit{Circular Polarization of Simulated Images of Black Holes}

\item Event Horizon Telescope Collaboration, , Akiyama, K., Alberdi, A., et al., incl.\ \textbf{Joshi, A. V.} 2024, \href{https://ui.adsabs.harvard.edu/abs/2024ApJ...964L..26E}{ApJL, 964, L26}\\\textit{First Sagittarius A* Event Horizon Telescope Results. VIII. Physical Interpretation of the Polarized Ring}

\item Event Horizon Telescope Collaboration, , Akiyama, K., Alberdi, A., et al., incl.\ \textbf{Joshi, A. V.} 2023, \href{https://ui.adsabs.harvard.edu/abs/2023ApJ...957L..20E}{ApJL, 957, L20}\\\textit{First M87 Event Horizon Telescope Results. IX. Detection of Near-horizon Circular Polarization}

\item Conroy, N. S., Bauböck, M., Dhruv, V., et al., incl.\ \textbf{Joshi, A. V.} 2023, \href{https://ui.adsabs.harvard.edu/abs/2023ApJ...951...46C}{ApJ, 951, 46}\\\textit{Rotation in Event Horizon Telescope Movies}

\item \textbf{Joshi, A. V.}, Rosofsky, S. G., Haas, R., \& Huerta, E. A. 2023, \href{https://ui.adsabs.harvard.edu/abs/2023PhRvD.107f4038J}{PhRvD, 107, 064038}\\\textit{Numerical relativity higher order gravitational waveforms of eccentric, spinning, nonprecessing binary black hole mergers}

\item Event Horizon Telescope Collaboration, , Akiyama, K., Alberdi, A., et al., incl.\ \textbf{Joshi, A. V.} 2022, \href{https://ui.adsabs.harvard.edu/abs/2022ApJ...930L..16E}{ApJL, 930, L16}\\\textit{First Sagittarius A* Event Horizon Telescope Results. V. Testing Astrophysical Models of the Galactic Center Black Hole}

\item Event Horizon Telescope Collaboration, , Akiyama, K., Alberdi, A., et al., incl.\ \textbf{Joshi, A. V.} 2022, \href{https://ui.adsabs.harvard.edu/abs/2022ApJ...930L..12E}{ApJL, 930, L12}\\\textit{First Sagittarius A* Event Horizon Telescope Results. I. The Shadow of the Supermassive Black Hole in the Center of the Milky Way}

\item Wong, G. N., Prather, B. S., Dhruv, V., et al., incl.\ \textbf{Joshi, A. V.} 2022, \href{https://ui.adsabs.harvard.edu/abs/2022ApJS..259...64W}{ApJS, 259, 64}\\\textit{PATOKA\@: Simulating Electromagnetic Observables of Black Hole Accretion}

\item Marszewski, A., Prather, B. S., \textbf{Joshi, A. V.}, Pandya, A., et al. 2021, \href{https://ui.adsabs.harvard.edu/abs/2021ApJ...921...17M}{ApJ, 921, 17}\\\textit{Updated Transfer Coefficients for Magnetized Plasmas}

\item Pandya, A., Chandra, M., \textbf{Joshi, A. V.}, \& Gammie, C. F. 2018, \href{https://ui.adsabs.harvard.edu/abs/2018ApJ...868...13P}{ApJ, 868, 13}\\\textit{Numerical Evaluation of the Relativistic Magnetized Plasma Susceptibility Tensor and Faraday Rotation Coefficients}


\end{etaremune}


\section*{Mentoring}
Illinois Guidance for Physics Students (GPS) \hfill 2023--present\\
\indentpar{Mentored three undergraduate students about physics research, grad school applications and general academic guidance.}

Computational Astrophysics Group Mentoring \hfill 2021--present\\
\indentpar{Mentored four undergraduate students on their own projects within Charles Gammie's research group.}

UIUC Young Scholars Program \hfill Summer 2018\\
\indentpar{Mentored a high school student over the summer on a simulation visualization project.}

\section*{Teaching}
Teaching Assistant, University of Illinois Urbana-Champaign

PHYS 598, Computational Physics and Astrophysics \hfill Fall 2022, Fall 2024

PHYS 214, University Physics: Quantum Physics \hfill Summer 2020

PHYS 212, University Physics: Electricity \& Magnetism \hfill Spring 2020\\
\indentpar{Ranked excellent teacher by students.}

PHYS 102, College Physics: E\&M and Modern Physics \hfill Fall 2019\\
\indentpar{Ranked excellent teacher by students.}

\section*{Outreach and Volunteering}
Education Justice Project (EJP), Writing and Math Program Tutor \hfill 2023--present\\
\indentpar{Tutored students enrolled within the EJP initiative at the Danville Correctional Center in math, writing and general black hole astrophysics concepts.}

Speaker, Satkama High School, Hyderabad, India \hfill Jan 2024\\
\indentpar{Engaged with students from 6th to 10th grades by presenting concepts of horizon-scale black hole astrophysics with demos and animations.}

Lab Presenter, Conference for Undergraduate Women in Physics (CUWiP) \hfill Jan 2023\\
\indentpar{Gave a talk to participants of the conference about Charles Gammie's research group activities and EHT science.}

Black hole science demo, University Primary School \hfill May 2022\\
\indentpar{Interacted with students from 2nd to 5th grades about black holes and EHT science. Coordinated an outdoor activity for the students.}


\ifdefined\withtalks{}
\section*{Talks}

\else\fi

\ifdefined\withpress{}
\section*{Press}
CHTC, “Junior Researchers Advance Black Hole Research with OSPool Open Capacity,” April 29, 2024, \href{https://chtc.cs.wisc.edu/eht-story.html}{https://chtc.cs.wisc.edu/eht-story.html}.

Grainger Engineering Office of Marketing and Communications, “A Supermassive Black Hole’s Strong Magnetic Fields Are Revealed in a New Light,” \href{https://icasu.illinois.edu/news/EHT-2023}{https://icasu.illinois.edu/news/EHT-2023}.

Grainger Engineering Office of Marketing and Communications, “Gammie Group at Illinois Physics Contributes to First-Ever Image of Supermassive Black Hole at Milky Way’s Galactic Center,” \href{https://physics.illinois.edu/news/EHT-Images\_Sag-A-Star}{https://physics.illinois.edu/news/EHT-Images\_Sag-A-Star}.

\else\fi

\ifdefined\withcourses{}
\section*{Graduate coursework}

\else\fi
\end{document}
